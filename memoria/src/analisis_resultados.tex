\section{Análisis de resultados}

Para comenzar, de cara a realizar el análisis de resultados, se han realizado resúmenes de los resultados obtenidos, utilizando las medias de las ejecuciones y los mejores resultados obtenidos.

Utilizando dichos resúmenes para interpretar de una forma más sencilla los resultados, se discutirán distintos aspectos de estos:

\begin{enumerate}
	\item Diferencias según la función de ajuste utilizada.
	\item Comparación entre los distintos algoritmos utilizados.
	\item Diferencias según la técnica de sobremuestreo utilizada.
	\item Comparación con otras técnicas del estado del arte.
\end{enumerate}

Además, por último se realizará un análisis del uso de características, de cara a encontrar que atributos se han utilizado más y así estudiar si alguno de estos atributos no es necesario.

En este análisis, para referirse a los algoritmos y sus ejecuciones se seguirá la siguiente nomenclatura: \texttt{<ALG>-<OVERSAMPLING>-<FITNESS>}. Donde, \texttt{ALG} será el algoritmo de Programación Genética utilizado:

\begin{itemize}
	\item BJOR: Algoritmo de Bjorczuk.
	\item FALCO: Algoritmo de Falco.
	\item TAN: Algoritmo de Tan.
\end{itemize}

\texttt{OVERSAMPLING} será la técnica de balanceo de clases utilizada:

\begin{itemize}
	\item ROS: Sobremuestro aleatorio (Random Over Sampling).
	\item SMOTE: El algoritmo SMOTE.
	\item BLSMOTE: El algoritmo BorderlineSMOTE.
\end{itemize}

\texttt{FITNESS} será la función de ajuste utilizada durante el entrenamiento del algoritmo:

\begin{itemize}
	\item OMAE.
	\item MMAE.
	\item AMAE.
\end{itemize}

De forma que, por ejemplo, TAN-SMOTE-OMAE quiere decir que se refiere a los resultados obtenidos por el algoritmo de Tan, utilizando el conjunto de datos con sobremuestro usando SMOTE, y entrenado con OMAE como función de ajuste.

Se ha decidido no incluir los resultados con la función de ajuste original de cada algoritmo ya que se ha visto que con dicha función de ajuste los algoritmos no son capaces de encontrar soluciones competentes al problema, e incluirlos en este análisis solo lo extendería innecesariamente.

\subsection{Resumen de los resultados}

Vamos a comenzar con una tabla a modo de resumen general de todas las ejecuciones. Esta tabla contiene los resultados en media de todos los algoritmos, técnicas de sobremuestro y funciones de ajuste utilizadas.

% Please add the following required packages to your document preamble:
% \usepackage{multirow}
% \usepackage{graphicx}
\begin{table}[H]
\centering
\resizebox{\textwidth}{!}{%
\begin{tabular}{|ccccccc|}
\hline
\multicolumn{7}{|c|}{\textbf{Tabla con un resumen de los resultados en media obtenidos por todas las ejecuciones.}}                                                                                                                                                               \\ \hline
\multicolumn{1}{|c|}{\multirow{2}{*}{\textbf{}}}  & \multicolumn{3}{c|}{\textbf{OMAE}}                                                                                        & \multicolumn{3}{c|}{\textbf{Accuracy}}                                                            \\ \cline{2-7}
\multicolumn{1}{|c|}{}                            & \multicolumn{1}{c|}{\textbf{Training}} & \multicolumn{1}{c|}{\textbf{Validacion}} & \multicolumn{1}{c|}{\textbf{Test}}    & \multicolumn{1}{c|}{\textbf{Training}} & \multicolumn{1}{c|}{\textbf{Validacion}} & \textbf{Test} \\ \hline
\multicolumn{1}{|c|}{\textbf{BJOR-ROS-OMAE}}      & \multicolumn{1}{c|}{1,8573}          & \multicolumn{1}{c|}{1,9112}            & \multicolumn{1}{c|}{1,9691}         & \multicolumn{1}{c|}{0,2216}           & \multicolumn{1}{c|}{0,2080}             & 0,0927       \\ \hline
\multicolumn{1}{|c|}{\textbf{BJOR-SMOTE-OMAE}}    & \multicolumn{1}{c|}{1,7362}           & \multicolumn{1}{c|}{1,6805}             & \multicolumn{1}{c|}{1,7972}          & \multicolumn{1}{c|}{0,2544}           & \multicolumn{1}{c|}{0,2518}              & 0,0968       \\ \hline
\multicolumn{1}{|c|}{\textbf{BJOR-BLSMOTE-OMAE}}  & \multicolumn{1}{c|}{2,1888}           & \multicolumn{1}{c|}{2,1557}             & \multicolumn{1}{c|}{2,1569}          & \multicolumn{1}{c|}{0,1796}            & \multicolumn{1}{c|}{0,1715}             & 0,1135       \\ \hline \hline
\multicolumn{1}{|c|}{\textbf{BJOR-ROS-MMAE}}      & \multicolumn{1}{c|}{2,0320}          & \multicolumn{1}{c|}{2,0497}            & \multicolumn{1}{c|}{2,3816}         & \multicolumn{1}{c|}{0,1938}           & \multicolumn{1}{c|}{0,1816}             & 0,0885       \\ \hline
\multicolumn{1}{|c|}{\textbf{BJOR-SMOTE-MMAE}}    & \multicolumn{1}{c|}{2,6121}           & \multicolumn{1}{c|}{2,6252}             & \multicolumn{1}{c|}{2,4246}          & \multicolumn{1}{c|}{0,2247}           & \multicolumn{1}{c|}{0,2219}             & 0,1125        \\ \hline
\multicolumn{1}{|c|}{\textbf{BJOR-BLSMOTE-MMAE}}  & \multicolumn{1}{c|}{2,6709}            & \multicolumn{1}{c|}{2,6776}             & \multicolumn{1}{c|}{2,4960}          & \multicolumn{1}{c|}{0,1381}           & \multicolumn{1}{c|}{0,1255}             & 0,1010       \\ \hline \hline
\multicolumn{1}{|c|}{\textbf{BJOR-ROS-AMAE}}      & \multicolumn{1}{c|}{2,1410}           & \multicolumn{1}{c|}{2,1938}             & \multicolumn{1}{c|}{2,0212}          & \multicolumn{1}{c|}{0,2336}           & \multicolumn{1}{c|}{0,2212}             & 0,1864       \\ \hline
\multicolumn{1}{|c|}{\textbf{BJOR-SMOTE-AMAE}}    & \multicolumn{1}{c|}{2,4353}           & \multicolumn{1}{c|}{2,4740}             & \multicolumn{1}{c|}{2,3815}           & \multicolumn{1}{c|}{0,1595}           & \multicolumn{1}{c|}{0,1579}              & 0,0854       \\ \hline
\multicolumn{1}{|c|}{\textbf{BJOR-BLSMOTE-AMAE}}  & \multicolumn{1}{c|}{2,4353}           & \multicolumn{1}{c|}{2,4740}             & \multicolumn{1}{c|}{2,3815}           & \multicolumn{1}{c|}{0,1595}           & \multicolumn{1}{c|}{0,1579}              & 0,0854       \\ \hline \hline
\multicolumn{1}{|c|}{\textbf{FALCO-ROS-OMAE}}     & \multicolumn{1}{c|}{1,8201}           & \multicolumn{1}{c|}{1,8741}             & \multicolumn{1}{c|}{1,7541}          & \multicolumn{1}{c|}{0,3258}            & \multicolumn{1}{c|}{0,3118}             & 0,2218       \\ \hline
\multicolumn{1}{|c|}{\textbf{FALCO-SMOTE-OMAE}}   & \multicolumn{1}{c|}{1,9551}           & \multicolumn{1}{c|}{1,9139}             & \multicolumn{1}{c|}{1,8394}          & \multicolumn{1}{c|}{0,3214}            & \multicolumn{1}{c|}{0,3212}             & 0,2531       \\ \hline
\multicolumn{1}{|c|}{\textbf{FALCO-BLSMOTE-OMAE}} & \multicolumn{1}{c|}{1,7912}           & \multicolumn{1}{c|}{1,8748}             & \multicolumn{1}{c|}{2,033}            & \multicolumn{1}{c|}{0,3634}           & \multicolumn{1}{c|}{0,3428}              & 0,1708       \\ \hline \hline
\multicolumn{1}{|c|}{\textbf{TAN-ROS-OMAE}}       & \multicolumn{1}{c|}{1,5948}           & \multicolumn{1}{c|}{1,5746}             & \multicolumn{1}{c|}{\textbf{1,5856}} & \multicolumn{1}{c|}{0,2566}           & \multicolumn{1}{c|}{0,2568}             & 0,1896        \\ \hline
\multicolumn{1}{|c|}{\textbf{TAN-SMOTE-OMAE}}     & \multicolumn{1}{c|}{1,4658}           & \multicolumn{1}{c|}{1,4912}             & \multicolumn{1}{c|}{1,6448}          & \multicolumn{1}{c|}{0,2922}           & \multicolumn{1}{c|}{0,2719}             & 0,2145       \\ \hline
\multicolumn{1}{|c|}{\textbf{TAN-BLSMOTE-OMAE}}   & \multicolumn{1}{c|}{1,4595}           & \multicolumn{1}{c|}{1,5016}             & \multicolumn{1}{c|}{1,7807}          & \multicolumn{1}{c|}{0,2950}           & \multicolumn{1}{c|}{0,2780}             & 0,1822       \\ \hline
\end{tabular}%
}
\caption{Tabla resumen con todas las ejecuciones realizadas.}
\end{table}

Aunque esta tabla sigue siendo demasiado grande como para intentar utilizarla en todas las secciones, si que agrupa todas las ejecuciones realizadas a lo largo de la experimentación. En los siguientes apartados se utilizarán tablas obtenidas a partir de esta tabla general para realizar las comparativas entre las distintas técnicas utilizadas.

\subsection{Importancia de la función de ajuste en este problema}

Para comenzar con este análisis de resultados vamos a comentar la importancia de la función de ajuste de los algoritmos a la hora de resolver el problema.

Uno de los problemas encontrados al realizar la experimentación ha sido los malos resultados iniciales obtenidos por los tres algoritmos utilizados, independientemente de si utilizaban un enfoque Michigan, un enfoque híbrido, si escogían una única regla por clase, o si escogían más reglas por clase. Tras comprobar el comportamiento de los tres algoritmos y analizar sus resultados, la conclusión es que no era capaz de aprender reglas para un problema complejo con los medios que contaba el algoritmo, por lo que se decidió cambiar la forma de explorar el espacio de búsqueda, adaptando dicha búsqueda para un problema de clasificación ordinal.

Este fue el motivo de que se estudiaran otras métricas como función de ajuste en los algoritmos, en concreto funciones utilizadas en problemas de clasificación ordinal:

\begin{itemize}
	\item OMAE.
	\item AMAE.
	\item MMAE.
\end{itemize}

Y aunque las dos últimas solo se han podido aplicar al algoritmo de Bjorczuk (ya que el algoritmo de Tan y el de Falco aprenden las clases por separado), el comportamiento ha sido el siguiente, separando los resultados por función de ajuste utilizada:

% Please add the following required packages to your document preamble:
% \usepackage{multirow}
% \usepackage{graphicx}
\begin{table}[H]
\centering
\resizebox{\textwidth}{!}{%
\begin{tabular}{|ccccccc|}
\hline
\multicolumn{7}{|c|}{\textbf{\begin{tabular}[c]{@{}c@{}}Tabla con un resumen de los resultados en media obtenidos por \\ las ejecuciones con la función de ajuste original.\end{tabular}}}                                                                                     \\ \hline
\multicolumn{1}{|c|}{\multirow{2}{*}{}}           & \multicolumn{3}{c|}{\textbf{OMAE}}                                                                                     & \multicolumn{3}{c|}{\textbf{Accuracy}}                                                            \\ \cline{2-7}
\multicolumn{1}{|c|}{}                            & \multicolumn{1}{c|}{\textbf{Training}} & \multicolumn{1}{c|}{\textbf{Validacion}} & \multicolumn{1}{c|}{\textbf{Test}} & \multicolumn{1}{c|}{\textbf{Training}} & \multicolumn{1}{c|}{\textbf{Validacion}} & \textbf{Test} \\ \hline
\multicolumn{1}{|c|}{\textbf{BJOR-ROS-ORIG}}      & \multicolumn{1}{c|}{3,2632}          & \multicolumn{1}{c|}{3,2596}             & \multicolumn{1}{c|}{3,2685}      & \multicolumn{1}{c|}{0,1007}           & \multicolumn{1}{c|}{0,1010}             & 0,1573        \\ \hline
\multicolumn{1}{|c|}{\textbf{BJOR-SMOTE-ORIG}}    & \multicolumn{1}{c|}{2,6564}           & \multicolumn{1}{c|}{2,7200}             & \multicolumn{1}{c|}{2,7419}        & \multicolumn{1}{c|}{0,1947}           & \multicolumn{1}{c|}{0,1820}             & 0,0895       \\ \hline
\multicolumn{1}{|c|}{\textbf{BJOR-BLSMOTE-ORIG}}  & \multicolumn{1}{c|}{3,3474}           & \multicolumn{1}{c|}{3,3498}              & \multicolumn{1}{c|}{3,3594}       & \multicolumn{1}{c|}{0,1002}           & \multicolumn{1}{c|}{0,0985}             & 0,1854        \\ \hline
\multicolumn{1}{|c|}{\textbf{FALCO-ROS-ORIG}}     & \multicolumn{1}{c|}{4,3313}           & \multicolumn{1}{c|}{4,3385}             & \multicolumn{1}{c|}{4,3895}        & \multicolumn{1}{c|}{0,1733}           & \multicolumn{1}{c|}{0,1716}             & 0,0510       \\ \hline
\multicolumn{1}{|c|}{\textbf{FALCO-SMOTE-ORIG}}   & \multicolumn{1}{c|}{4,2311}           & \multicolumn{1}{c|}{4,2311}             & \multicolumn{1}{c|}{4,2923}       & \multicolumn{1}{c|}{0,2010}           & \multicolumn{1}{c|}{0,2010}             & 0,0677        \\ \hline
\multicolumn{1}{|c|}{\textbf{FALCO-BLSMOTE-ORIG}} & \multicolumn{1}{c|}{4,3344}           & \multicolumn{1}{c|}{4,3502}             & \multicolumn{1}{c|}{4,2869}       & \multicolumn{1}{c|}{0,1492}           & \multicolumn{1}{c|}{0,1410}             & 0,0687       \\ \hline
\multicolumn{1}{|c|}{\textbf{TAN-ROS-ORIG}}       & \multicolumn{1}{c|}{3,9962}           & \multicolumn{1}{c|}{3,9788}             & \multicolumn{1}{c|}{4,0425}       & \multicolumn{1}{c|}{0,0951}            & \multicolumn{1}{c|}{0,0959}              & 0,1781       \\ \hline
\multicolumn{1}{|c|}{\textbf{TAN-SMOTE-ORIG}}     & \multicolumn{1}{c|}{3,9326}           & \multicolumn{1}{c|}{3,9661}             & \multicolumn{1}{c|}{4,0411}       & \multicolumn{1}{c|}{0,1098}            & \multicolumn{1}{c|}{0,1039}             & 0,3354       \\ \hline
\multicolumn{1}{|c|}{\textbf{TAN-BLSMOTE-ORIG}}   & \multicolumn{1}{c|}{4,4829}           & \multicolumn{1}{c|}{4,4909}             & \multicolumn{1}{c|}{4,44}          & \multicolumn{1}{c|}{0,1001}            & \multicolumn{1}{c|}{0,0999}             & 0,3594        \\ \hline
\end{tabular}%
}
\caption{Tabla resumen con los resultados usando la función de ajuste propuesta en cada algoritmo.}\label{resumenDEFECTO}
\end{table}


% Please add the following required packages to your document preamble:
% \usepackage{multirow}
% \usepackage{graphicx}
\begin{table}[H]
\centering
\resizebox{\textwidth}{!}{%
\begin{tabular}{|ccccccc|}
\hline
\multicolumn{7}{|c|}{\textbf{\begin{tabular}[c]{@{}c@{}}Tabla con un resumen de los resultados en media obtenidos por \\ las ejecuciones con la función de ajuste OMAE.\end{tabular}}}                                                                                     \\ \hline
\multicolumn{1}{|c|}{\multirow{2}{*}{}}           & \multicolumn{3}{c|}{\textbf{OMAE}}                                                                                     & \multicolumn{3}{c|}{\textbf{Accuracy}}                                                            \\ \cline{2-7}
\multicolumn{1}{|c|}{}                            & \multicolumn{1}{c|}{\textbf{Training}} & \multicolumn{1}{c|}{\textbf{Validacion}} & \multicolumn{1}{c|}{\textbf{Test}} & \multicolumn{1}{c|}{\textbf{Training}} & \multicolumn{1}{c|}{\textbf{Validacion}} & \textbf{Test} \\ \hline
\multicolumn{1}{|c|}{\textbf{BJOR-ROS-OMAE}}      & \multicolumn{1}{c|}{1,8573}          & \multicolumn{1}{c|}{1,9112}            & \multicolumn{1}{c|}{1,9691}      & \multicolumn{1}{c|}{0,2216}           & \multicolumn{1}{c|}{0,2080}             & 0,0927       \\ \hline
\multicolumn{1}{|c|}{\textbf{BJOR-SMOTE-OMAE}}    & \multicolumn{1}{c|}{1,7362}           & \multicolumn{1}{c|}{1,6805}             & \multicolumn{1}{c|}{1,7972}       & \multicolumn{1}{c|}{0,2544}           & \multicolumn{1}{c|}{0,2518}              & 0,0968       \\ \hline
\multicolumn{1}{|c|}{\textbf{BJOR-BLSMOTE-OMAE}}  & \multicolumn{1}{c|}{2,1888}           & \multicolumn{1}{c|}{2,1557}             & \multicolumn{1}{c|}{2,1569}       & \multicolumn{1}{c|}{0,1796}            & \multicolumn{1}{c|}{0,1715}             & 0,1135       \\ \hline
\multicolumn{1}{|c|}{\textbf{FALCO-ROS-OMAE}}     & \multicolumn{1}{c|}{1,8201}           & \multicolumn{1}{c|}{1,8741}             & \multicolumn{1}{c|}{1,7541}       & \multicolumn{1}{c|}{0,3258}            & \multicolumn{1}{c|}{0,3118}             & 0,2218       \\ \hline
\multicolumn{1}{|c|}{\textbf{FALCO-SMOTE-OMAE}}   & \multicolumn{1}{c|}{1,9551}           & \multicolumn{1}{c|}{1,9139}             & \multicolumn{1}{c|}{1,8394}       & \multicolumn{1}{c|}{0,3214}            & \multicolumn{1}{c|}{0,3212}             & 0,2531       \\ \hline
\multicolumn{1}{|c|}{\textbf{FALCO-BLSMOTE-OMAE}} & \multicolumn{1}{c|}{1,7912}           & \multicolumn{1}{c|}{1,8748}             & \multicolumn{1}{c|}{2,033}         & \multicolumn{1}{c|}{0,3634}           & \multicolumn{1}{c|}{0,3428}              & 0,1708       \\ \hline
\multicolumn{1}{|c|}{\textbf{TAN-ROS-OMAE}}       & \multicolumn{1}{c|}{1,5948}           & \multicolumn{1}{c|}{1,5746}             & \multicolumn{1}{c|}{1,5856}       & \multicolumn{1}{c|}{0,2566}           & \multicolumn{1}{c|}{0,2568}             & 0,1896        \\ \hline
\multicolumn{1}{|c|}{\textbf{TAN-SMOTE-OMAE}}     & \multicolumn{1}{c|}{1,4658}           & \multicolumn{1}{c|}{1,4912}             & \multicolumn{1}{c|}{1,6448}       & \multicolumn{1}{c|}{0,2922}           & \multicolumn{1}{c|}{0,2719}             & 0,2145       \\ \hline
\multicolumn{1}{|c|}{\textbf{TAN-BLSMOTE-OMAE}}   & \multicolumn{1}{c|}{1,4595}           & \multicolumn{1}{c|}{1,5016}             & \multicolumn{1}{c|}{1,7807}       & \multicolumn{1}{c|}{0,2950}           & \multicolumn{1}{c|}{0,2780}             & 0,1822       \\ \hline
\end{tabular}%
}
\caption{Tabla resumen con los resultados usando la función de ajuste OMAE.}\label{resumenOMAE}
\end{table}


% Please add the following required packages to your document preamble:
% \usepackage{multirow}
% \usepackage{graphicx}
\begin{table}[H]
\centering
\resizebox{\textwidth}{!}{%
\begin{tabular}{|ccccccc|}
\hline
\multicolumn{7}{|c|}{\textbf{\begin{tabular}[c]{@{}c@{}}Tabla con un resumen de los resultados en media obtenidos por \\ las ejecuciones con la función de ajuste MMAE.\end{tabular}}}                                                                                    \\ \hline
\multicolumn{1}{|c|}{\multirow{2}{*}{}}          & \multicolumn{3}{c|}{\textbf{OMAE}}                                                                                     & \multicolumn{3}{c|}{\textbf{Accuracy}}                                                            \\ \cline{2-7}
\multicolumn{1}{|c|}{}                           & \multicolumn{1}{c|}{\textbf{Training}} & \multicolumn{1}{c|}{\textbf{Validacion}} & \multicolumn{1}{c|}{\textbf{Test}} & \multicolumn{1}{c|}{\textbf{Training}} & \multicolumn{1}{c|}{\textbf{Validacion}} & \textbf{Test} \\ \hline
\multicolumn{1}{|c|}{\textbf{BJOR-ROS-MMAE}}     & \multicolumn{1}{c|}{2,0320}          & \multicolumn{1}{c|}{2,0497}            & \multicolumn{1}{c|}{2,3816}      & \multicolumn{1}{c|}{0,1938}           & \multicolumn{1}{c|}{0,1816}             & 0,0885       \\ \hline
\multicolumn{1}{|c|}{\textbf{BJOR-SMOTE-MMAE}}   & \multicolumn{1}{c|}{2,6121}           & \multicolumn{1}{c|}{2,6252}             & \multicolumn{1}{c|}{2,4246}       & \multicolumn{1}{c|}{0,2247}           & \multicolumn{1}{c|}{0,2219}             & 0,1125        \\ \hline
\multicolumn{1}{|c|}{\textbf{BJOR-BLSMOTE-MMAE}} & \multicolumn{1}{c|}{2,6709}            & \multicolumn{1}{c|}{2,6776}             & \multicolumn{1}{c|}{2,4960}       & \multicolumn{1}{c|}{0,1381}           & \multicolumn{1}{c|}{0,1255}             & 0,1010       \\ \hline
\end{tabular}%
}
\caption{Tabla resumen con los resultados usando la función de ajuste MMAE.}\label{resumenMMAE}
\end{table}


% Please add the following required packages to your document preamble:
% \usepackage{multirow}
% \usepackage{graphicx}
\begin{table}[H]
\centering
\resizebox{\textwidth}{!}{%
\begin{tabular}{|ccccccc|}
\hline
\multicolumn{7}{|c|}{\textbf{\begin{tabular}[c]{@{}c@{}}Tabla con un resumen de los resultados en media obtenidos por \\ las ejecuciones con la función de ajuste AMAE.\end{tabular}}}                                                                                    \\ \hline
\multicolumn{1}{|c|}{\multirow{2}{*}{}}          & \multicolumn{3}{c|}{\textbf{OMAE}}                                                                                     & \multicolumn{3}{c|}{\textbf{Accuracy}}                                                            \\ \cline{2-7}
\multicolumn{1}{|c|}{}                           & \multicolumn{1}{c|}{\textbf{Training}} & \multicolumn{1}{c|}{\textbf{Validacion}} & \multicolumn{1}{c|}{\textbf{Test}} & \multicolumn{1}{c|}{\textbf{Training}} & \multicolumn{1}{c|}{\textbf{Validacion}} & \textbf{Test} \\ \hline
\multicolumn{1}{|c|}{\textbf{BJOR-ROS-AMAE}}     & \multicolumn{1}{c|}{2,1410}           & \multicolumn{1}{c|}{2,1938}             & \multicolumn{1}{c|}{2,0212}       & \multicolumn{1}{c|}{0,2336}           & \multicolumn{1}{c|}{0,2212}             & 0,1864       \\ \hline
\multicolumn{1}{|c|}{\textbf{BJOR-SMOTE-AMAE}}   & \multicolumn{1}{c|}{2,4353}           & \multicolumn{1}{c|}{2,4740}             & \multicolumn{1}{c|}{2,3815}        & \multicolumn{1}{c|}{0,1595}           & \multicolumn{1}{c|}{0,1579}              & 0,0854       \\ \hline
\multicolumn{1}{|c|}{\textbf{BJOR-BLSMOTE-AMAE}} & \multicolumn{1}{c|}{2,4353}           & \multicolumn{1}{c|}{2,4740}             & \multicolumn{1}{c|}{2,3815}        & \multicolumn{1}{c|}{0,1595}           & \multicolumn{1}{c|}{0,1579}              & 0,0854       \\ \hline
\end{tabular}%
}
\caption{Tabla resumen con los resultados usando la función de ajuste AMAE.}\label{resumenAMAE}
\end{table}


Como podemos ver en los resultados de la tabla \ref{resumenDEFECTO} en comparación con todas las demás ejecuciones usando métricas para clasificación ordinal (\ref{resumenOMAE}, \ref{resumenMMAE} y \ref{resumenAMAE}), y ya vimos en las matrices de confusión de los resultados, en ninguno de los algoritmos da buenos resultados la función de ajuste propuesta, independientemente del algoritmo concreto utilizado, siendo los resultados significativamente peores, y no siendo capaz ninguno de ellos de realizar predicciones complejas.

Por otro lado, comparando las tres métricas de clasificación ordinal para el algoritmo de Bjorczuk, la elección clara se trata de OMAE, ya que en los conjuntos de test y validación consigue unos resultados mejores que usando tanto MMAE como AMAE, llegando a veces a mejorar el OMAE en un $0.6$.

El haber realizado esta modificación de la función de ajuste, y que este cambio implique unas diferencias tan grandes en la ejecución del algoritmo nos lleva a la conclusión de la importancia de conocer el problema que estamos manejando. No se trata simplemente de conocer el dominio del problema y como se comportan los datos que utilizamos, también es importante adaptar los algoritmos al problema al que nos enfrentamos de cara a que la búsqueda por el espacio de soluciones sea coherente con el problema y lo que se busca.

Por simplicidad y debido a que claramente OMAE consigue unos mejores resultados, en las siguientes secciones nos limitaremos a utilizar los resultados obtenidos por los experimentos que han utilizado OMAE.


\subsection{Comparación entre los algoritmos utilizados}

En este apartado vamos a comparar los resultados de los tres algoritmos utilizados. En este caso, además de las métricas utilizadas, otra de las cosas que tenemos que tener en cuenta es la simplicidad de la solución, ya que aunque los algoritmos de Bjorczuk y de Falco como máximo tienen once reglas (una por clase y la regla por defecto), el algoritmo de Tan, como hemos visto en algunos resultados, puede generar clasificadores con un conjunto de reglas mucho más grande.


Vamos a comenzar observando las métricas de evaluación:

% Please add the following required packages to your document preamble:
% \usepackage{multirow}
% \usepackage{graphicx}
\begin{table}[H]
\centering
\resizebox{\textwidth}{!}{%
\begin{tabular}{|ccccccc|}
\hline
\multicolumn{7}{|c|}{\textbf{\begin{tabular}[c]{@{}c@{}}Tabla con un resumen de los resultados en media obtenidos por \\ las ejecuciones del algoritmo de Bjorczuk.\end{tabular}}}                                                                                            \\ \hline
\multicolumn{1}{|c|}{\multirow{2}{*}{}}          & \multicolumn{3}{c|}{\textbf{OMAE}}                                                                                     & \multicolumn{3}{c|}{\textbf{Accuracy}}                                                            \\ \cline{2-7}
\multicolumn{1}{|c|}{}                           & \multicolumn{1}{c|}{\textbf{Training}} & \multicolumn{1}{c|}{\textbf{Validacion}} & \multicolumn{1}{c|}{\textbf{Test}} & \multicolumn{1}{c|}{\textbf{Training}} & \multicolumn{1}{c|}{\textbf{Validacion}} & \textbf{Test} \\ \hline
\multicolumn{1}{|c|}{\textbf{BJOR-ROS-OMAE}}     & \multicolumn{1}{c|}{1,8573}          & \multicolumn{1}{c|}{1,9112}            & \multicolumn{1}{c|}{1,9691}      & \multicolumn{1}{c|}{0,2216}           & \multicolumn{1}{c|}{0,2080}             & 0,0927       \\ \hline
\multicolumn{1}{|c|}{\textbf{BJOR-SMOTE-OMAE}}   & \multicolumn{1}{c|}{1,7362}           & \multicolumn{1}{c|}{1,6805}             & \multicolumn{1}{c|}{1,7972}       & \multicolumn{1}{c|}{0,2544}           & \multicolumn{1}{c|}{0,2518}              & 0,0968       \\ \hline
\multicolumn{1}{|c|}{\textbf{BJOR-BLSMOTE-OMAE}} & \multicolumn{1}{c|}{2,1888}           & \multicolumn{1}{c|}{2,1557}             & \multicolumn{1}{c|}{2,1569}       & \multicolumn{1}{c|}{0,1796}            & \multicolumn{1}{c|}{0,1715}             & 0,1135       \\ \hline
\end{tabular}%
}
\caption{Tabla resumen con los resultados del algoritmo de Bjorczuk y la función de ajuste OMAE.}\label{resumenBjorczukOMAE}
\end{table}

% Please add the following required packages to your document preamble:
% \usepackage{multirow}
% \usepackage{graphicx}
\begin{table}[H]
\centering
\resizebox{\textwidth}{!}{%
\begin{tabular}{|ccccccc|}
\hline
\multicolumn{7}{|c|}{\textbf{\begin{tabular}[c]{@{}c@{}}Tabla con un resumen de los resultados en media obtenidos por \\ las ejecuciones del algoritmo de Falco.\end{tabular}}}                                                                                                \\ \hline
\multicolumn{1}{|c|}{\multirow{2}{*}{}}           & \multicolumn{3}{c|}{\textbf{OMAE}}                                                                                     & \multicolumn{3}{c|}{\textbf{Accuracy}}                                                            \\ \cline{2-7}
\multicolumn{1}{|c|}{}                            & \multicolumn{1}{c|}{\textbf{Training}} & \multicolumn{1}{c|}{\textbf{Validacion}} & \multicolumn{1}{c|}{\textbf{Test}} & \multicolumn{1}{c|}{\textbf{Training}} & \multicolumn{1}{c|}{\textbf{Validacion}} & \textbf{Test} \\ \hline
\multicolumn{1}{|c|}{\textbf{FALCO-ROS-OMAE}}     & \multicolumn{1}{c|}{1,8201}           & \multicolumn{1}{c|}{1,8741}             & \multicolumn{1}{c|}{1,7541}       & \multicolumn{1}{c|}{0,3258}            & \multicolumn{1}{c|}{0,3118}             & 0,2218       \\ \hline
\multicolumn{1}{|c|}{\textbf{FALCO-SMOTE-OMAE}}   & \multicolumn{1}{c|}{1,9551}           & \multicolumn{1}{c|}{1,9139}             & \multicolumn{1}{c|}{1,8394}       & \multicolumn{1}{c|}{0,3214}            & \multicolumn{1}{c|}{0,3212}             & 0,2531       \\ \hline
\multicolumn{1}{|c|}{\textbf{FALCO-BLSMOTE-OMAE}} & \multicolumn{1}{c|}{1,7912}           & \multicolumn{1}{c|}{1,8748}             & \multicolumn{1}{c|}{2,033}         & \multicolumn{1}{c|}{0,3634}           & \multicolumn{1}{c|}{0,3428}              & 0,1708       \\ \hline
\end{tabular}%
}
\caption{Tabla resumen con los resultados del algoritmo de Falco y la función de ajuste OMAE.}\label{resumenFalcoOMAE}

\end{table}


% Please add the following required packages to your document preamble:
% \usepackage{multirow}
% \usepackage{graphicx}
\begin{table}[]
\centering
\resizebox{\textwidth}{!}{%
\begin{tabular}{|ccccccc|}
\hline
\multicolumn{7}{|c|}{\textbf{\begin{tabular}[c]{@{}c@{}}Tabla con un resumen de los resultados en media obtenidos por \\ las ejecuciones del algoritmo de Tan.\end{tabular}}}                                                                                                \\ \hline
\multicolumn{1}{|c|}{\multirow{2}{*}{}}         & \multicolumn{3}{c|}{\textbf{OMAE}}                                                                                     & \multicolumn{3}{c|}{\textbf{Accuracy}}                                                            \\ \cline{2-7}
\multicolumn{1}{|c|}{}                          & \multicolumn{1}{c|}{\textbf{Training}} & \multicolumn{1}{c|}{\textbf{Validacion}} & \multicolumn{1}{c|}{\textbf{Test}} & \multicolumn{1}{c|}{\textbf{Training}} & \multicolumn{1}{c|}{\textbf{Validacion}} & \textbf{Test} \\ \hline
\multicolumn{1}{|c|}{\textbf{TAN-ROS-OMAE}}     & \multicolumn{1}{c|}{1,5948}           & \multicolumn{1}{c|}{1,5746}             & \multicolumn{1}{c|}{1,5856}       & \multicolumn{1}{c|}{0,2566}           & \multicolumn{1}{c|}{0,2568}             & 0,1896        \\ \hline
\multicolumn{1}{|c|}{\textbf{TAN-SMOTE-OMAE}}   & \multicolumn{1}{c|}{1,4658}           & \multicolumn{1}{c|}{1,4912}             & \multicolumn{1}{c|}{1,6448}       & \multicolumn{1}{c|}{0,2922}           & \multicolumn{1}{c|}{0,2719}             & 0,2145       \\ \hline
\multicolumn{1}{|c|}{\textbf{TAN-BLSMOTE-OMAE}} & \multicolumn{1}{c|}{1,4595}           & \multicolumn{1}{c|}{1,5016}             & \multicolumn{1}{c|}{1,7807}       & \multicolumn{1}{c|}{0,2950}           & \multicolumn{1}{c|}{0,2780}             & 0,1822       \\ \hline
\end{tabular}%
}
\caption{Tabla resumen con los resultados del algoritmo de Tan y la función de ajuste OMAE.}\label{resumenTanOMAE}
\end{table}

Vemos en las tablas \ref{resumenBjorczukOMAE} y \ref{resumenFalcoOMAE} que los resultados de Bojarczuk y Falco son similares, y las diferencias en los resultados pueden deberse simplemente a la aleatoriedad de las ejecuciones, sin embargo vemos en el cuadro \ref{resumenTanOMAE} que el algoritmo de Tan si es más competitivo que los otros. Sin embargo, otro detalle que tenemos que observar, es el número de reglas obtenidas por el clasificador:

% Please add the following required packages to your document preamble:
% \usepackage{graphicx}
\begin{table}[]
\centering
\begin{tabular}{|c|c|}
\hline
               & \textbf{\begin{tabular}[c]{@{}c@{}}N. Reglas en promedio\\  del modelo\end{tabular}} \\ \hline
\textbf{BJOR}  & 11                                                                                   \\ \hline
\textbf{FALCO} & 11                                                                                   \\ \hline
\textbf{TAN}   & 39                                                                                   \\ \hline
\end{tabular}%
\caption{Tabla resumen con el promedio de reglas obtenidas por cada modelo.}\label{resumenNReglas}
\end{table}

Si recordamos como funcionaban los distintos algoritmos, las puestas de Bjorczuk y Falco se limitaban a escoger una única regla por clase, mientras que el algoritmo de Tan escogía todas las reglas de la población elitista eliminando las repetidas. Este detalle hace que la solución encontrada por el algoritmo de Tan sea mucho más compleja y con una cantidad de reglas mucho mayor.

Aunque es cierto que los resultados en el conjunto de test son mejores (con una diferencia de $0.17$ en OMAE si comparamos el mejor resultado en media del algoritmo de Tan, con el mejor resultado en media obtenido por el algoritmo de Falco), una diferencia tan baja en OMAE es posible que no justifique la complejidad añadida de las soluciones propuestas.

\newpage

\subsection{Discusión sobre las técnicas de balanceo de clases aplicadas}

Por otra parte, para este problema también nos interesa tener una buena forma de tratar el desbalanceo de clases. Como se ha comentado con anterioridad, se han tomado tres enfoques distintos, dos de ellos basados en introducir instancias sintéticas:

\begin{enumerate}
	\item Sobremuestreo aleatorio: Repetir instancias de las clases minoritarias hasta que la distribución de observaciones por clase sea uniforme.
	\item SMOTE: Algoritmo para generar instancias sintéticas usando técnicas basadas en distancias.
	\item Borderline-SMOTE: Variante de SMOTE, centrada en generar instancias sintéticas en las fronteras de decisión.
\end{enumerate}

Vamos a analizar los resultados obtenidos por los tres algoritmos con los distintos enfoques para resolver este problema:

% Please add the following required packages to your document preamble:
% \usepackage{multirow}
% \usepackage{graphicx}
\begin{table}[H]
\centering
\resizebox{\textwidth}{!}{%
\begin{tabular}{|ccccccc|}
\hline
\multicolumn{7}{|c|}{\textbf{\begin{tabular}[c]{@{}c@{}}Tabla con un resumen de los resultados en media obtenidos por \\ con el conjunto de datos al que se ha aplicado sobremuestro aleatorio.\end{tabular}}}                                                             \\ \hline
\multicolumn{1}{|c|}{\multirow{2}{*}{}}       & \multicolumn{3}{c|}{\textbf{OMAE}}                                                                                     & \multicolumn{3}{c|}{\textbf{Accuracy}}                                                            \\ \cline{2-7}
\multicolumn{1}{|c|}{}                        & \multicolumn{1}{c|}{\textbf{Training}} & \multicolumn{1}{c|}{\textbf{Validacion}} & \multicolumn{1}{c|}{\textbf{Test}} & \multicolumn{1}{c|}{\textbf{Training}} & \multicolumn{1}{c|}{\textbf{Validacion}} & \textbf{Test} \\ \hline
\multicolumn{1}{|c|}{\textbf{BJOR-ROS-OMAE}}     & \multicolumn{1}{c|}{1,8573}          & \multicolumn{1}{c|}{1,9112}            & \multicolumn{1}{c|}{1,9691}      & \multicolumn{1}{c|}{0,2216}           & \multicolumn{1}{c|}{0,2080}             & 0,0927       \\ \hline
\multicolumn{1}{|c|}{\textbf{FALCO-ROS-OMAE}}    & \multicolumn{1}{c|}{1,8201}           & \multicolumn{1}{c|}{1,8741}             & \multicolumn{1}{c|}{1,7541}       & \multicolumn{1}{c|}{0,3258}            & \multicolumn{1}{c|}{0,3118}             & 0,2218       \\ \hline
\multicolumn{1}{|c|}{\textbf{TAN-ROS-OMAE}}      & \multicolumn{1}{c|}{1,5948}           & \multicolumn{1}{c|}{1,5746}             & \multicolumn{1}{c|}{1,5856}       & \multicolumn{1}{c|}{0,2566}           & \multicolumn{1}{c|}{0,2568}             & 0,1896        \\ \hline
\end{tabular}%
}
\caption{Tabla resumen con los resultados de aplicar sobremuestreo aleatorio.}\label{resumenROS}
\end{table}

% Please add the following required packages to your document preamble:
% \usepackage{multirow}
% \usepackage{graphicx}
\begin{table}[H]
\centering
\resizebox{\textwidth}{!}{%
\begin{tabular}{|ccccccc|}
\hline
\multicolumn{7}{|c|}{\textbf{\begin{tabular}[c]{@{}c@{}}Tabla con un resumen de los resultados en media obtenidos por \\ con el conjunto de datos al que se ha aplicado SMOTE.\end{tabular}}}                                                                                \\ \hline
\multicolumn{1}{|c|}{\multirow{2}{*}{}}         & \multicolumn{3}{c|}{\textbf{OMAE}}                                                                                     & \multicolumn{3}{c|}{\textbf{Accuracy}}                                                            \\ \cline{2-7}
\multicolumn{1}{|c|}{}                          & \multicolumn{1}{c|}{\textbf{Training}} & \multicolumn{1}{c|}{\textbf{Validacion}} & \multicolumn{1}{c|}{\textbf{Test}} & \multicolumn{1}{c|}{\textbf{Training}} & \multicolumn{1}{c|}{\textbf{Validacion}} & \textbf{Test} \\ \hline
\multicolumn{1}{|c|}{\textbf{BJOR-SMOTE-OMAE}}   & \multicolumn{1}{c|}{1,7362}           & \multicolumn{1}{c|}{1,6805}             & \multicolumn{1}{c|}{1,7972}       & \multicolumn{1}{c|}{0,2544}           & \multicolumn{1}{c|}{0,2518}              & 0,0968       \\ \hline
\multicolumn{1}{|c|}{\textbf{FALCO-SMOTE-OMAE}}  & \multicolumn{1}{c|}{1,9551}           & \multicolumn{1}{c|}{1,9139}             & \multicolumn{1}{c|}{1,8394}       & \multicolumn{1}{c|}{0,3214}            & \multicolumn{1}{c|}{0,3212}             & 0,2531       \\ \hline
\multicolumn{1}{|c|}{\textbf{TAN-SMOTE-OMAE}}    & \multicolumn{1}{c|}{1,4658}           & \multicolumn{1}{c|}{1,4912}             & \multicolumn{1}{c|}{1,6448}       & \multicolumn{1}{c|}{0,2922}           & \multicolumn{1}{c|}{0,2719}             & 0,2145       \\ \hline
\end{tabular}%
}
\caption{Tabla resumen con los resultados de aplicar SMOTE.}\label{resumenSMOTE}
\end{table}

% Please add the following required packages to your document preamble:
% \usepackage{multirow}
% \usepackage{graphicx}
\begin{table}[H]
\centering
\resizebox{\textwidth}{!}{%
\begin{tabular}{|ccccccc|}
\hline
\multicolumn{7}{|c|}{\textbf{\begin{tabular}[c]{@{}c@{}}Tabla con un resumen de los resultados en media obtenidos por \\ con el conjunto de datos al que se ha aplicado BL-SMOTE.\end{tabular}}}                                                                               \\ \hline
\multicolumn{1}{|c|}{\multirow{2}{*}{}}           & \multicolumn{3}{c|}{\textbf{OMAE}}                                                                                     & \multicolumn{3}{c|}{\textbf{Accuracy}}                                                            \\ \cline{2-7}
\multicolumn{1}{|c|}{}                            & \multicolumn{1}{c|}{\textbf{Training}} & \multicolumn{1}{c|}{\textbf{Validacion}} & \multicolumn{1}{c|}{\textbf{Test}} & \multicolumn{1}{c|}{\textbf{Training}} & \multicolumn{1}{c|}{\textbf{Validacion}} & \textbf{Test} \\ \hline
\multicolumn{1}{|c|}{\textbf{BJOR-BLSMOTE-OMAE}}  & \multicolumn{1}{c|}{2,1888}           & \multicolumn{1}{c|}{2,1557}             & \multicolumn{1}{c|}{2,1569}       & \multicolumn{1}{c|}{0,1796}            & \multicolumn{1}{c|}{0,1715}             & 0,1135       \\ \hline
\multicolumn{1}{|c|}{\textbf{FALCO-BLSMOTE-OMAE}} & \multicolumn{1}{c|}{1,7912}           & \multicolumn{1}{c|}{1,8748}             & \multicolumn{1}{c|}{2,033}         & \multicolumn{1}{c|}{0,3634}           & \multicolumn{1}{c|}{0,3428}              & 0,1708       \\ \hline
\multicolumn{1}{|c|}{\textbf{TAN-BLSMOTE-OMAE}}   & \multicolumn{1}{c|}{1,4595}           & \multicolumn{1}{c|}{1,5016}             & \multicolumn{1}{c|}{1,7807}       & \multicolumn{1}{c|}{0,2950}           & \multicolumn{1}{c|}{0,2780}             & 0,1822       \\ \hline
\end{tabular}%
}
\caption{Tabla resumen con los resultados de aplicar BL-SMOTE.}\label{resumenBLSMOTE}
\end{table}



En este caso podemos ver comportamientos bastante distintos. Comenzando por el algoritmo de Bjorczuk, aplicar SMOTE en lugar de sobremuestro aleatorio ha mejorado los resultados notablemente, sin embargo, Borderline-SMOTE no ha funcionado. Por otro lado, tanto para el algoritmo de Falco como para el de Tan el uso de SMOTE y Borderline-SMOTE han empeorado los resultados obtenidos por sobremuestro aleatorio, siendo peores los resultados con Borderline-SMOTE.

Vamos a comenzar comentando el caso más claro, Borderline-SMOTE. El utilizar esta técnica para resolver el balanceo de clases ha dado peores resultados en todos los casos. Este comportamiento puede ser debido a que, al introducir datos sintéticos en las fronteras de decisión, se estén creando fronteras de decisión que no corresponden con las reales, y por lo tanto se cometa un mayor error.

Por otro lado, con respecto al comportamiento de los algoritmo al utilizar SMOTE, vemos dos comportamientos distintos, el del algoritmo de Bjorczuk, que mejora los resultados, y los algoritmos de Falco y de Tan, donde los resultados son peores. En este caso este comportamiento se puede explicar sabiendo como funcionan los algoritmos. El algoritmo de Bjorczuk, al trabajar con todas las clases en la misma ejecución, es capaz de tener en cuenta el solapamiento entre todas las clases y manejarlo en las reglas generadas, mientras que los algoritmo con un enfoque Michigan no son capaces de tener esto en cuenta, y generan reglas con un mayor solape.

Este estudio, además de poder comparar los resultados obtenidos, nos deja claro que la principal dificultad a la hora de obtener un buen resultado son las fronteras de decisión. Por un lado, si nos centramos en ellas, como en el caso de BL-SMOTE, es probable que las instancias sintéticas introduzcan ruido en lugar de aportar a la solución, debido a la falta de instancias originales con las que contamos, mientras que por otro lado tenemos que tener conocer lo suficiente los algoritmos que se utilizan como para saber si en realidad estamos ayudando al algoritmo a aprender o por el contrario estamos introduciéndole ruido.

\newpage

\subsection{Comparación con las técnicas del estado del arte}

De cara a realizar una comparación con las técnicas del estado del arte se ha escogido la ejecución con mejores resultados de cada uno de los algoritmos utilizados:

\begin{itemize}
	\item BJOR-SMOTE-OMAE.
	\item FALCO-ROS-OMAE.
	\item TAN-ROS-OMAE.
\end{itemize}

Se utilizarán los resultados de \cite{NSLVOrdAge}, ya que se tratan de los mejores resultados hasta el momento para el problema, también utilizando un enfoque de clasificación en las diez fases propuestas por Todd. En su propuesta se utilizan los siguientes algoritmos:

\begin{itemize}
	\item NSLVOrd\cite{NSLVOrd}: Descubrimiento de reglas para clasificación ordinal utilizando algoritmos evolutivos y la estrategia de cubrimiento secuencial de las observaciones.
	\item RandomForest: RF100 en la tabla de resultados, al contar con 100 árboles de forma interna.
	\item Redes neuronales profundas: DNN en la tabla comparativa.
\end{itemize}

Además, también utilizan tanto sobremuestreo aleatorio como BL-SMOTE en sus resultados.


% Please add the following required packages to your document preamble:
% \usepackage{graphicx}
\begin{table}[H]
\centering
\begin{tabular}{|cccc|}
\hline
\multicolumn{4}{|c|}{\textbf{\begin{tabular}[c]{@{}c@{}}Tabla comparativa de nuestros resultados\\  con el estado del arte\end{tabular}}}          \\ \hline
\multicolumn{1}{|c|}{\textbf{}}                & \multicolumn{1}{c|}{\textbf{OMAE}}  & \multicolumn{1}{c|}{\textbf{Accuracy}} & \textbf{N. reglas} \\ \hline
\multicolumn{1}{|c|}{\textbf{BJOR-SMOTE-OMAE}} & \multicolumn{1}{c|}{1,7972}        & \multicolumn{1}{c|}{0,0968}           & \textbf{11}        \\ \hline
\multicolumn{1}{|c|}{\textbf{FALCO-ROS-OMAE}}  & \multicolumn{1}{c|}{1,7541}        & \multicolumn{1}{c|}{0,2218}           & \textbf{11}        \\ \hline
\multicolumn{1}{|c|}{\textbf{TAN-ROS-OMAE}}    & \multicolumn{1}{c|}{1,5856}        & \multicolumn{1}{c|}{0,1896}            & 39                 \\ \hline
\multicolumn{1}{|c|}{\textbf{NSLVOrd-ROS}}     & \multicolumn{1}{c|}{1,337}          & \multicolumn{1}{c|}{0,373}             & 39                 \\ \hline
\multicolumn{1}{|c|}{\textbf{RF100-ROS}}       & \multicolumn{1}{c|}{1,27}           & \multicolumn{1}{c|}{0,432}             & 11593              \\ \hline
\multicolumn{1}{|c|}{\textbf{DNN-ROS}}         & \multicolumn{1}{c|}{1,148}          & \multicolumn{1}{c|}{0,34}              & -                  \\ \hline
\multicolumn{1}{|c|}{\textbf{NSLVOrd-BLSMOTE}} & \multicolumn{1}{c|}{1,285}          & \multicolumn{1}{c|}{0,398}             & 35                 \\ \hline
\multicolumn{1}{|c|}{\textbf{RF100-BLSMOTE}}   & \multicolumn{1}{c|}{1,146}          & \multicolumn{1}{c|}{\textbf{0,439}}    & 9433               \\ \hline
\multicolumn{1}{|c|}{\textbf{DNN-BLSMOTE}}     & \multicolumn{1}{c|}{\textbf{1,063}} & \multicolumn{1}{c|}{0,363}             & -                  \\ \hline
\end{tabular}%
\end{table}


Aunque las técnicas utilizadas en este trabajo no llegan a alcanzar los resultados del estado del arte en precisión y OMAE, nuestra propuesta es una solución mucho más sencilla e interpretable, ya que contamos con solo una regla por clase y la regla por defecto, siendo un sistema mucho más simple. Por otro lado, en comparación con el estado del arte, en este caso los resultados del algoritmo de Tan ya no son tan interesantes, ya que vemos como NSLVOrd cuenta con un conjunto de reglas de un tamaño similar, y obtiene unos resultados bastante mejores, y aunque los resultados del algoritmo de Bjorczuk y de Falco son peores, siguen contando con el valor añadido de ser soluciones lo más simples posibles para resolver el problema, lo que en este ámbito donde buscamos un método de aprendizaje automático explicable es de gran interés.

Además de los resultados de NSLVOrd, también contamos con los resultados de Random Forest y redes neuronales profundas, que como vemos son los que mejores métricas obtienen tanto en OMAE como en tasa de acierto, sin embargo, como suele ocurrir en la gran mayoría de ocasiones, estos modelos con resultados muy buenos no son nada interpretables. Random Forest ha obtenido un conjunto de $9.433$ reglas, algo que no es viable analizar por un experto y poder obtener conocimiento de dichos resultados, mientras que el modelo de red neuronal profunda, aunque existen formas de analizarlos y explicar el modelo a posteriori como se comenta en \cite{XAI}, sigue siendo un modelo de caja negra.

Con esta comparación podemos ver que, aunque no conseguimos unos resultados competitivos con el estado del arte en OMAE y tasa de acierto, nuestra propuesta es la más competitiva con respecto a la complejidad del modelo final, manteniendo un OMAE bastante bueno, lo que puede ser de gran ayuda en ámbitos donde se busca que estos modelos sean una ayuda a la toma de decisiones y una fuente donde extraer conocimiento, como es el caso de este problema, en lugar de reemplazar al experto.

\newpage

\subsection{Análisis de uso de características}

Otro de los estudios de interés dentro de este problema es si realmente las diez características de Todd son buenas y necesarias para obtener un resultado. Por este motivo, se ha realizado un análisis de uso de características. Usando los resultados obtenidos por las tres mejores ejecuciones, una por algoritmo, se ha hecho un conteo de las características utilizadas:

% Please add the following required packages to your document preamble:
% \usepackage{graphicx}
\begin{table}[H]
\centering
\resizebox{\textwidth}{!}{%
\begin{tabular}{|ccccccccccc|}
\hline
\multicolumn{11}{|c|}{\textbf{Uso de características de las mejores ejecuciones escogidas}}                                                                                                                                                                                                                                                                                                                                                     \\ \hline
\multicolumn{1}{|c|}{\textbf{}}                     & \multicolumn{1}{c|}{\textbf{$x_0$}} & \multicolumn{1}{c|}{\textbf{$x_1$}} & \multicolumn{1}{c|}{\textbf{$x_2$}} & \multicolumn{1}{c|}{\textbf{$x_3$}} & \multicolumn{1}{c|}{\textbf{$x_4$}} & \multicolumn{1}{c|}{\textbf{$x_5$*}} & \multicolumn{1}{c|}{\textbf{$x_6$}} & \multicolumn{1}{c|}{\textbf{$x_7$}} & \multicolumn{1}{c|}{\textbf{$x_8$}} & \textbf{N. total de caraterísticas} \\ \hline
\multicolumn{1}{|c|}{\textbf{BJOR-SMOTE-OMAE}}      & \multicolumn{1}{c|}{4}              & \multicolumn{1}{c|}{5}              & \multicolumn{1}{c|}{2}              & \multicolumn{1}{c|}{3}              & \multicolumn{1}{c|}{3}              & \multicolumn{1}{c|}{0}              & \multicolumn{1}{c|}{0}              & \multicolumn{1}{c|}{4}              & \multicolumn{1}{c|}{3}              & 24                                  \\ \hline
\multicolumn{1}{|c|}{\textbf{FALCO-ROS-OMAE}}       & \multicolumn{1}{c|}{12}             & \multicolumn{1}{c|}{8}              & \multicolumn{1}{c|}{0}              & \multicolumn{1}{c|}{2}              & \multicolumn{1}{c|}{7}              & \multicolumn{1}{c|}{0}              & \multicolumn{1}{c|}{1}              & \multicolumn{1}{c|}{8}              & \multicolumn{1}{c|}{13}             & 51                                  \\ \hline
\multicolumn{1}{|c|}{\textbf{TAN-ROS-OMAE}}         & \multicolumn{1}{c|}{13}             & \multicolumn{1}{c|}{31}             & \multicolumn{1}{c|}{19}             & \multicolumn{1}{c|}{16}             & \multicolumn{1}{c|}{20}             & \multicolumn{1}{c|}{0}              & \multicolumn{1}{c|}{19}             & \multicolumn{1}{c|}{22}             & \multicolumn{1}{c|}{21}             & 161                                 \\ \hline
\multicolumn{1}{|c|}{\textbf{Total}}                & \multicolumn{1}{c|}{29}             & \multicolumn{1}{c|}{44}             & \multicolumn{1}{c|}{21}             & \multicolumn{1}{c|}{21}             & \multicolumn{1}{c|}{30}             & \multicolumn{1}{c|}{0}              & \multicolumn{1}{c|}{20}             & \multicolumn{1}{c|}{34}             & \multicolumn{1}{c|}{37}             & 236                                 \\ \hline
\multicolumn{1}{|c|}{\textbf{\% respecto al total}} & \multicolumn{1}{c|}{12,29\%}        & \multicolumn{1}{c|}{18,64\%}        & \multicolumn{1}{c|}{8,90\%}         & \multicolumn{1}{c|}{8,90\%}         & \multicolumn{1}{c|}{12,71\%}        & \multicolumn{1}{c|}{0,00\%}         & \multicolumn{1}{c|}{8,47\%}         & \multicolumn{1}{c|}{14,41\%}        & \multicolumn{1}{c|}{15,68\%}        &                                     \\ \hline
\end{tabular}%
}
\caption{Uso de características de los clasificadores de las mejores ejecuciones.\\ \textbf{*} La variable $x_5$, \texttt{DorsalMargin}, fue eliminada en el preprocesamiento ya que solamente tomaba un valor.} \label{tablaUsoCaracteristicas}
\end{table}

Donde:

\begin{itemize}
	\item $x_0$ es la característica \texttt{ArticularFace}.
	\item $x_1$ es la característica \texttt{IrregularPorosity}.
	\item $x_2$ es la característica \texttt{UpperSymphysialExtremity}.
	\item $x_3$ es la característica \texttt{BonyNodule}.
	\item $x_4$ es la característica \texttt{LowerSymphysialExtremity}.
	\item $x_5$ es la característica \texttt{DorsalMargin}.
	\item $x_6$ es la característica \texttt{DorsalPlaeau}.
	\item $x_7$ es la característica \texttt{VentralBevel}.
	\item $x_8$ es la característica \texttt{VentralMargin}.
\end{itemize}

Lo primero que podemos observar en la tabla \ref{tablaUsoCaracteristicas} es que el algoritmo de Bjorczuk es el que menos veces utiliza las características en las reglas, tan solo 24 veces, mientras que el algoritmo de Falco duplica ese valor con 51 apariciones, y el algoritmo de Tan triplica el uso de Falco, con 161 apariciones. Era de esperar que el algoritmo de Tan usara más característica al contar con un conjunto de reglas mucho más grande.

En nuestros resultados la variable más usada es \texttt{IrregularPorosity}, seguida por \texttt{VentralMargin}, con porcentajes de aparición de más de un $15\%$ con respecto al total. Claramente podemos distinguir en dos grupos los predictores más y menos utilizados, ya que contamos con cuatro variables con menos de un $9\%$ de uso con respecto al total de características, un total del $26,27\%$, mientras que las otras cinco características cuentan con un uso de más del $12\$$, si contamos el total, un $73,73\%$. Es decir, solo la mitad de características son utilizadas son utilizadas casi tres de cada cuatro veces que se usa una característica.

Claramente en este problema hay características que son utilizadas con una mayor frecuencia, y que seguramente tengan una mayor relevancia en la toma de decisiones. Esto concuerda con lo discutido en el estado del arte, donde varias propuestas reducían el número de predictores utilizados de nueve a cinco, como en \cite{componentBased}.



\newpage
