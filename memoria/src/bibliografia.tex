\addcontentsline{toc}{section}{Referencias}

\begin{thebibliography}{9}


	\bibitem{todd}

	\href{https://onlinelibrary.wiley.com/doi/abs/10.1002/ajpa.1330030301}{T. W. Todd, “Age changes in the pubic bone,” American Journal of Physical Anthropology, vol. 3, no. 3, pp. 285–328, 1920.}



	\bibitem{XAI}

	\href{https://www.sciencedirect.com/science/article/pii/S1566253519308103?via%3Dihub}{Arrieta, A. B., Díaz-Rodríguez, N., Del Ser, J., Bennetot, A., Tabik, S., Barbado, A., ... \& Herrera, F. (2020). Explainable Artificial Intelligence (XAI): Concepts, taxonomies, opportunities and challenges toward responsible AI. Information Fusion, 58, 82-115.}


	\bibitem{laboratorioForenseUGR}

	Antropología Forense – LiveMetrics UGR. \url{https://livemetrics.ugr.es/laboratorio-singular/antropologia-forense/}

	\bibitem{sucheyBrooks}

	\href{https://link.springer.com/article/10.1007/BF02437238}{Brooks, S., \& Suchey, J. M. (1990). Skeletal age determination based on the os pubis: a comparison of the Acsádi-Nemeskéri and Suchey-Brooks methods. Human evolution, 5(3), 227-238.}

	\bibitem{modelandoHuesos3D}

	\href{https://onlinelibrary.wiley.com/doi/full/10.1111/1556-4029.12778}{Slice, D. E., \& Algee‐Hewitt, B. F. (2015). Modeling bone surface morphology: a fully quantitative method for age‐at‐death estimation using the pubic symphysis. Journal of forensic sciences, 60(4), 835-843.}

	\bibitem{mejoraModelandoHuesos3D}

	\href{https://onlinelibrary.wiley.com/doi/full/10.1002/ajpa.22797}{Stoyanova, D., Algee‐Hewitt, B. F., \& Slice, D. E. (2015). An enhanced computational method for age‐at‐death estimation based on the pubic symphysis using 3 D laser scans and thin plate splines. American journal of physical anthropology, 158(3), 431-440.}


	% \bibitem{segundaMejoraModelandoHuesos3D}
	%
	% \href{https://onlinelibrary.wiley.com/doi/full/10.1111/1556-4029.13439}{Stoyanova, D. K., Algee‐Hewitt, B. F., Kim, J., \& Slice, D. E. (2017). A computational framework for age‐at‐death estimation from the skeleton: surface and outline analysis of 3D laser scans of the adult pubic symphysis. Journal of forensic sciences, 62(6), 1434-1444.}


	\bibitem{componentBased}

	\href{https://www.sciencedirect.com/science/article/pii/S0379073815003254}{Dudzik, B., \& Langley, N. R. (2015). Estimating age from the pubic symphysis: A new component-based system. Forensic science international, 257, 98-105.}

	\bibitem{estimacionHuesosCadera}

	\href{https://www.sciencedirect.com/science/article/pii/S0379073818301440}{Kotěrová, A., Navega, D., Štepanovský, M., Buk, Z., Brůžek, J., \& Cunha, E. (2018). Age estimation of adult human remains from hip bones using advanced methods. Forensic science international, 287, 163-175.}


	\bibitem{fuzzyAgeEstimation}

	\href{https://ieeexplore.ieee.org/abstract/document/8015760}{Villar, P., Alemán, I., Castillo, L., Damas, S., \& Cordón, O. (2017, July). A first approach to a fuzzy classification system for age estimation based on the pubic bone. In 2017 IEEE International Conference on Fuzzy Systems (FUZZ-IEEE) (pp. 1-6). IEEE.}

	\bibitem{NSLVOrdAge}

	Gámez-Granados, J. C., Irurita, J., Pérez, R., González, A., Damas, S., Alemán, I. \& Cordón, O. Automating Todd’s Age Estimation Method from the Pubic Bone with Explainable Machine Learning. Sometido a revista.

	\bibitem{NSLVOrd}

	\href{https://www.sciencedirect.com/science/article/pii/S0888613X16300706}{Gámez, J. C., García, D., González, A., \& Pérez, R. (2016). Ordinal classification based on the sequential covering strategy. International Journal of Approximate Reasoning, 76, 96-110.}

	% \bibitem{analisisRegresionSimbolica}
	%
	% \href{https://link.springer.com/chapter/10.1007/978-3-642-56181-8_31}{Billard, L., \& Diday, E. (2002). Symbolic regression analysis. In Classification, Clustering, and Data Analysis (pp. 281-288). Springer, Berlin, Heidelberg.}

	\bibitem{primeraPropuestaMcKern}

	McKern, T. W., \& Stewart, T. D. (1957). Skeletal age changes in young American males: analysed from the standpoint of age identification (Vol. 45). Headquarters, Quartermaster Research \& Development Command.

	\bibitem{propuestaGilbert}

	\href{https://onlinelibrary.wiley.com/doi/abs/10.1002/ajpa.1330380109}{Gilbert, B. M., \& McKern, T. W. (1973). A method for aging the female os pubis. American Journal of Physical Anthropology, 38(1), 31-38.}

	\bibitem{primerTrabajoGilbert}

	Hanihara, K., \& Suzuki, T. (1978). Estimation of age from the pubic symphysis by means of multiple regression analysis. American Journal of Physical Anthropology, 48(2), 233-239. \url{https://onlinelibrary.wiley.com/doi/abs/10.1002/ajpa.1330480218}

	\bibitem{estudioComparandoGilbertTodd}

	Sinha, A., \& Gupta, V. (1995). A study on estimation of age from pubic symphysis. Forensic science international, 75(1), 73-78. \url{https://www.sciencedirect.com/science/article/pii/037907389501772B}

	% \bibitem{recopilacionMcKern}
	%
	% McKern, T. W. (1976). Sexual dimorphism in the maturation of the human public symphysis. Giles and Friedlaender, 433-450.

	\bibitem{primerDENDRAL}

	\href{https://www.researchgate.net/profile/Bruce-Buchanan/publication/23865744_On_generality_and_problem_solving_A_case_study_using_the_DENDRAL_program/links/0c96052e1e6f7e6a57000000/On-generality-and-problem-solving-A-case-study-using-the-DENDRAL-program.pdf}{Feigenbaum, E. A., Buchanan, B. G., \& Lederberg, J. (1970). On generality and problem solving: A case study using the DENDRAL program.}

	\bibitem{tesisMYCIN}

	\href{https://apps.dtic.mil/sti/citations/ADA001373}{Shortliffe, E. H. (1974). MYCIN: a rule-based computer program for advising physicians regarding antimicrobial therapy selection. STANFORD UNIV CALIF DEPT OF COMPUTER SCIENCE.}

	\bibitem{propuestaCN2}

	\href{https://link.springer.com/article/10.1023/A:1022641700528}{Clark, P., \& Niblett, T. (1989). The CN2 induction algorithm. Machine learning, 3(4), 261-283.}

	\bibitem{propuestaRIPPER}

	\href{https://citeseerx.ist.psu.edu/viewdoc/download?doi=10.1.1.14.4483&rep=rep1&type=pdf}{Cohen, W. W. (1995). Learning to classify English text with ILP methods. Advances in inductive logic programming, 32, 124-143.}

	\bibitem{reglasUsandoArboles}

	\href{https://citeseerx.ist.psu.edu/viewdoc/download?doi=10.1.1.98.9054&rep=rep1&type=pdf}{Quinlan, J. R. (1987, August). Generating production rules from decision trees. In ijcai (Vol. 87, pp. 304-307).}

	\bibitem{reglasUsandoGeneticos}

	\href{https://ieeexplore.ieee.org/abstract/document/781916}{Cattral, R., Oppacher, F., \& Deugo, D. (1999, July). Rule acquisition with a genetic algorithm. In Proceedings of the 1999 Congress on Evolutionary Computation-CEC99 (Cat. No. 99TH8406) (Vol. 1, pp. 125-129). IEEE.}

	\bibitem{kozaGP}

	\href{https://mitpress.mit.edu/books/genetic-programming}{Koza, J. R., \& Koza, J. R. (1992). Genetic programming: on the programming of computers by means of natural selection (Vol. 1). MIT press.}

	\bibitem{PGgramaticas}

	\href{https://www.researchgate.net/profile/Pa-Whigham/publication/2450222_Grammatically-based_Genetic_Programming/links/55c3c89908aebc967df1b765/Grammatically-based-Genetic-Programming.pdf}{Whigham, P. A. (1995, July). Grammatically-based genetic programming. In Proceedings of the workshop on genetic programming: from theory to real-world applications (Vol. 16, No. 3, pp. 33-41).}

	\bibitem{PGcontrolRobots}

	\href{https://www.tandfonline.com/doi/abs/10.1080/10798587.1996.10750674}{Tunstel, E., \& Jamshidi, M. (1996). On genetic programming of fuzzy rule-based systems for intelligent control. Intelligent Automation \& Soft Computing, 2(3), 271-284.}

	\bibitem{trabajoChestPain}

	\href{https://ieeexplore.ieee.org/document/853480}{Bojarczuk, C. C., Lopes, H. S., \& Freitas, A. A. (2000). Genetic programming for knowledge discovery in chest-pain diagnosis. IEEE Engineering in Medicine and Biology Magazine, 19(4), 38-44.}

	\bibitem{GAPnichosFuzzyRules}

	\href{http://dmle.icmat.es/revistas/detalle.php?numero=1954}{Ramos, L. S., \& González, J. A. C. (2000). A niching scheme for steady state GA-P and its application to fuzzy rule based classifiers induction. Mathware and Soft Computing, 7(2-3), 337-350.}

	\bibitem{reglasDosDominiosMedicosComparacion}

	\href{https://www.sciencedirect.com/science/article/pii/S0933365704001058}{Tsakonas, A., Dounias, G., Jantzen, J., Axer, H., Bjerregaard, B., \& von Keyserlingk, D. G. (2004). Evolving rule-based systems in two medical domains using genetic programming. Artificial Intelligence in Medicine, 32(3), 195-216.}

	\bibitem{grammarBasedPG}

	\href{https://www.researchgate.net/profile/Sebastian-Ventura/publication/228722056_C_Induction_of_Classification_Rules_with_Grammar-Based_Genetic_Programming/links/09e41510a07a3e3047000000/C-Induction-of-Classification-Rules-with-Grammar-Based-Genetic-Programming.pdf}{Espejo, P. G., Romero, C., Ventura, S., \& Hervás, C. (2005, November). Induction of classification rules with grammar-based genetic programming. In Conference on Machine Intelligence (pp. 596-601).}

	\bibitem{mejorasPGreglas}

	\href{https://ieeexplore.ieee.org/abstract/document/4610073}{Weise, T., Zapf, M., \& Geihs, K. (2007, December). Rule-based genetic programming. In 2007 2nd Bio-Inspired Models of Network, Information and Computing Systems (pp. 8-15). IEEE.}

	\bibitem{primerGAP}

	\href{https://ieeexplore.ieee.org/stamp/stamp.jsp?tp=&arnumber=393137}{Howard, L. M., \& D'Angelo, D. J. (1995). The GA-P: A genetic algorithm and genetic programming hybrid. IEEE expert, 10(3), 11-15.}

	\bibitem{GAPredElectrica}

	\href{https://link.springer.com/article/10.1023/A:1008384630089}{Cordón, O., Herrera, F., \& Sánchez, L. (1999). Solving electrical distribution problems using hybrid evolutionary data analysis techniques. Applied Intelligence, 10(1), 5-24.}


	\bibitem{PGregresionSimbolica}

	\href{https://ieeexplore.ieee.org/document/889734}{Augusto, D. A., \& Barbosa, H. J. (2000, November). Symbolic regression via genetic programming. In Proceedings. Vol. 1. Sixth Brazilian Symposium on Neural Networks (pp. 173-178). IEEE.}

	\bibitem{GAPFormulasBooleanas}

	\href{https://upcommons.upc.edu/handle/2099/3586}{Cordón García, O., Moya Anegón, F. D., \& Zarco Fernández, C. (2000). A GA-P algorithm to automatically formulate extended Boolean queries for a fuzzy information retrieval system. Mathware \& soft computing. 2000 Vol. 7 Núm. 2 [-3].}

	% \bibitem{SMOGN}
	%
	% \href{http://proceedings.mlr.press/v74/branco17a/branco17a.pdf}{Branco, P., Torgo, L., \& Ribeiro, R. P. (2017, October). SMOGN: a pre-processing approach for imbalanced regression. In First international workshop on learning with imbalanced domains: Theory and applications (pp. 36-50). PMLR.}

	% \bibitem{oversamplingGussianNoise}
	%
	% \href{https://web.cs.dal.ca/~branco/PDFfiles/j3.pdf}{Branco, P., Ribeiro, R. P., \& Torgo, L. (2016). UBL: an R package for utility-based learning. arXiv preprint arXiv:1604.08079.}
	%

	\bibitem{revisionSMOTE}

	\href{https://www.jair.org/index.php/jair/article/view/11192}{Fernández, A., Garcia, S., Herrera, F., \& Chawla, N. V. (2018). SMOTE for learning from imbalanced data: progress and challenges, marking the 15-year anniversary. Journal of artificial intelligence research, 61, 863-905.}

	\bibitem{propuestaADASYN}

	\href{https://ieeexplore.ieee.org/abstract/document/4633969}{He, H., Bai, Y., Garcia, E. A., \& Li, S. (2008, June). ADASYN: Adaptive synthetic sampling approach for imbalanced learning. In 2008 IEEE international joint conference on neural networks (IEEE world congress on computational intelligence) (pp. 1322-1328). IEEE.}


	% \bibitem{SMOTER}
	%
	% \href{https://link.springer.com/chapter/10.1007/978-3-642-40669-0_33}{Torgo, L., Ribeiro, R. P., Pfahringer, B., \& Branco, P. (2013, September). Smote for regression. In Portuguese conference on artificial intelligence (pp. 378-389). Springer, Berlin, Heidelberg. }




	\bibitem{OpenMP}

	OpenMP ARB. «Home». OpenMP, \url{https://www.openmp.org}.

	\bibitem{gdb}

	GDB: The GNU Project Debugger. \url{https://www.gnu.org/software/gdb/}.

	\bibitem{valgrind}

	Valgrind Home. \url{https://valgrind.org/}.

	\bibitem{gtest}

	google/googletest. 2015. Google, 2021. GitHub, \url{https://github.com/google/googletest/blob/master/docs/primer.md}.

	\bibitem{git}

	Git. \url{https://git-scm.com/}.

	\bibitem{GHPages}

	«GitHub Pages». GitHub Pages, \url{https://pages.github.com/.}

	\bibitem{BL-SMOTE}

	\href{https://link.springer.com/chapter/10.1007/11538059_91}{Han, H., Wang, W. Y., \& Mao, B. H. (2005, August). Borderline-SMOTE: a new over-sampling method in imbalanced data sets learning. In International conference on intelligent computing (pp. 878-887). Springer, Berlin, Heidelberg.}

	\bibitem{SMOTE}

	\href{https://www.jair.org/index.php/jair/article/view/10302}{Chawla, N. V., Bowyer, K. W., Hall, L. O., \& Kegelmeyer, W. P. (2002). SMOTE: synthetic minority over-sampling technique. Journal of artificial intelligence research, 16, 321-357.}

	\bibitem{kNNSMOTEN}

	\href{https://dl.acm.org/doi/pdf/10.1145/7902.7906}{Stanfill, C., \& Waltz, D. (1986). Toward memory-based reasoning. Communications of the ACM, 29(12), 1213-1228.}

	\bibitem{imblearnBLSMOTE}

	BorderlineSMOTE — Imbalanced Learn Version 0.8.0. \url{https://imbalanced-learn.org/stable/references/generated/imblearn.over_sampling.BorderlineSMOTE.html}.


	\bibitem{historiaAlgoritmosEvolutivos}

	\href{https://www.researchgate.net/publication/216300863_A_history_of_evolutionary_computation}{Bäck, T., Fogel, D. B., \& Michalewicz, Z. (1997). Handbook of evolutionary computation. (pp.A2.3:1-12) Release, 97(1), B1.}

	\bibitem{estrategiasEvolucion}

	Rechenberg, I. (1971). Evolutionsstrategie: Optimierung technischer Systeme nach Prinzipien der biologischen Evolution. Dr.-Ing (Doctoral dissertation, Thesis, Technical University of Berlin, Department of Process Engineering).

	\bibitem{programacionEvolutiva}

	Fogel, L. J., Owens, A. J., \& Walsh, M. J. (1966). Artificial intelligence through simulated evolution.

	\bibitem{libroAlgoritmosGeneticos}

	Holland, J. H. (1975). Adaptation in natural and artificial systems. MIT press.

	% \bibitem{cruceBLXalfa}
	%
	% \href{https://www.sciencedirect.com/science/article/pii/B9780080948324500180}{Eshelman, L. J., \& Schaffer, J. D. (1993). Real-coded genetic algorithms and interval-schemata. In Foundations of genetic algorithms (Vol. 2, pp. 187-202). Elsevier.}
	%
	% \bibitem{cruceRadcliffe}
	%
	% Radcliffe, N. J. (1991). Equivalence class analysis of genetic algorithms. Complex systems, 5(2), 183-205.

	% \bibitem{mutacionMichalewicz}
	%
	% Michalewicz, Z. (1992). Genetic Algorithms + Data Structures = Evolutionaiy Programs (pp. 103).

	% \bibitem{GAPnichos}
	%
	% \href{https://digibuo.uniovi.es/dspace/handle/10651/30759}{Sánchez Ramos, L., \& Corrales González, J. A. (2000). A niching scheme for steady state GA-P and its application to fuzzy rule based classifiers induction. Mathware and Soft Computing, 7 (2-3).}

	\bibitem{propuesta5x2cv}

	Dietterich, T. G. (1998). Approximate statistical tests for comparing supervised classification learning algorithms. Neural computation, 10(7), 1895-1923. \url{https://direct.mit.edu/neco/article/10/7/1895/6224/Approximate-Statistical-Tests-for-Comparing}

	\bibitem{gplv3}

	The GNU General Public License v3.0 - GNU Project - Free Software Foundation. \url{https://www.gnu.org/licenses/gpl-3.0.en.html}.

	\bibitem{githubProyecto}

	Villegas Yeguas, Antonio David. advy99/algoritmos\_poblacion\_expresiones. 2021. GitHub, \url{https://github.com/advy99/algoritmos_poblacion_expresiones}.

	\bibitem{doxygen}

	Doxygen: Index. \url{https://www.doxygen.nl/index.html}.

	\bibitem{irisDataset}

	UCI Machine Learning Repository: Iris Data Set. \url{https://archive.ics.uci.edu/ml/datasets/Iris}.

\end{thebibliography}
